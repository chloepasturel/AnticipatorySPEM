%!TeX TS-program = Lualatex 
%!TeX encoding = UTF-8 Unicode 
%!TeX spellcheck = en-US

\documentclass[stdletter,8pt,dateno]{newlfm}%
%\usepackage{kpfonts}
\usepackage{etoolbox}

\makeatletter
\patchcmd{\@zfancyhead}{\fancy@reset}{\f@nch@reset}{}{}
\patchcmd{\@set@em@up}{\f@ncyolh}{\f@nch@olh}{}{}
\patchcmd{\@set@em@up}{\f@ncyolh}{\f@nch@olh}{}{}
\patchcmd{\@set@em@up}{\f@ncyorh}{\f@nch@orh}{}{}
\makeatother

\usepackage{url}
%-------definitions-----
\newcommand{\AuthorA}{Vacher}%
\newcommand{\AuthorB}{Meso}%
\newcommand{\AuthorC}{Perrinet}%
\newcommand{\AuthorD}{Peyr\'e}%
\newcommand{\FirstNameA}{Jonathan}%
\newcommand{\FirstNameB}{Andrew Isaac}%
\newcommand{\FirstNameC}{Laurent U.}%
\newcommand{\FirstNameD}{Gabriel}%
%\newcommand{\Institute}{Institut de Neurosciences de la Timone, CNRS / Aix-Marseille Universit\'e}%
\newcommand{\Institute}{D\'epartement de Math\'ematiques et Applications,\\ \'Ecole Normale Sup\'erieure,}%
%\newcommand{\Address}{27, Bd. Jean Moulin, 13385 Marseille Cedex 5, France}%
\newcommand{\Address}{45 rue d’Ulm, \\ F-75230 Paris Cedex 05,\\ France }%
\newcommand{\Website}{http://invibe.net/LaurentPerrinet}%
\newcommand{\EmailA}{jonathan.vacher@ens.fr}%
\newcommand{\EmailC}{Laurent.Perrinet@univ-amu.fr}%
%\newcommand{\Title}{Motion-based position coding as a model for flash-lag effect}%
%\newcommand{\Title}{The flash-lag effect is a motion-based position shift}%
\newcommand{\Title}{Bayesian Modeling of Motion Perception using Dynamical Stochastic Textures}%
\newcommand{\Abstract}{%
}%
\newcommand{\Keywords}{predictive coding, motion coherency, flash-lag effect, neural delays, diagonal model, motion extrapolation, probabilistic models}%
\newcommand{\SignificanceStatement}{

tata

}
%--------------------------
%\newcommand{\Journal}{Journal of Vision}%
%\newcommand{\Journal}{PLoS Computational Biology}%
\newcommand{\Journal}{Neural Computation}%
\widowpenalty=1000
\clubpenalty=1000

\newsavebox{\Lpalmb} \sbox{\Lpalmb}{\parbox[t]{1.75in}{\includegraphics[width=.25\textwidth]{ens.jpg}\includegraphics[width=.2\textwidth]{int.png}\includegraphics[width=.12\textwidth]{unic.png}}}%
%\makelth{Homea}{\Lheader{\usebox{\Lpalms}}}%
%
\Lheader{\usebox{\Lpalmb}}

\newlfmP{headermarginskip=2pt}
\newlfmP{sigsize=2pt}
\newlfmP{dateskipafter=2pt}
%\newlfmP{addrfromphone}
\newlfmP{addrfromemail}
\PhrPhone{Phone}
\PhrEmail{Email}

\namefrom{
%\AuthorA\ , \FirstNameA , 
%\AuthorB\ , \FirstNameB  and 
\FirstNameA\ \AuthorA\ 
}
\addrfrom{%
\FirstNameA\ \AuthorA\ \\
\Institute \\
%\AuthorB\\[6pt]
    \Address\\[6pt]
%\AuthorA\\[6pt]
%%    \AddressA
%    \LongAddressA
}
%        \phonefrom{\PhoneA}
\emailfrom{\EmailA\\[6pt]
    \today
    }

\addrto{%
}

\greetto{
To the editorial board member of \emph{\Journal},%
%To Whom It May Concern,
%Dear XXX
}
\closeline{Sincerely,}

\begin{document}
\begin{newlfm}
%5. Cover Letter
%Submissions should be accompanied by a brief covering letter from the corresponding author including full postal address, telephone number and e-mail address. This letter should contain two (100-word or shorter) summaries: a concise paragraph to the editor indicating the scientific grounds why the paper should be considered for a topical, interdisciplinary journal rather than for a single-discipline or archival journal; and a separate, 100-word summary of the paper's appeal to a popular (non-scientific) audience.

Please consider the following manuscript ``{\it \Title }'' that we submit for publication in \Journal . All the authors have been involved with the work, have approved the manuscript and agreed to its submission.

Hereby, we submit our revised manuscript, entitled ``{\it \Title }'' ,  along  with  our  response  to  the  reviewers  and  corresponding  supplementary  information.  


% TO UPDATE = short description
The present study details the complete formulation of a generative
model of artificial textures intended to probe visual perception. It is first derived in a set of
axiomatic steps constrained by biological plausibility. We then extend previous contributions
by detailing three equivalent formulations of the Gaussian dynamic texture
model. Our model enables real-time on-the-fly texture synthesis using time-discretized autoregressive
processes. We use the model to probe speed perception in humans psychophysically using zoom-like changes
in stimulus spatial frequency content. The data replicates previous findings that relative perceived speed is positively biased by spatial
frequency increments. The effect cannot be fully accounted for by previous models,
but a parametric Bayesian model successfully accounts for the perceptual bias.
%
Our  response  to  the  reviewers  includes  the  point-by-point  responses  to  each  issue  raised  by  
both reviewers. The reviewers’ comments were helpful, and we feel that the manuscript is substantially improved.  
We have ensured that our manuscript complies with all formatting and editorial requirements.  


% AM: I think this section should go after the description of the science
% 'We ensure that n' - i think this is unnecessary, also 'a long version should be elaborated on below... '
None of the material has been published or is under consideration for
publication in other journals. This work constitutes a comprehensive extension to the conference paper referenced as \textit{Biologically Inspired Dynamic Textures for Probing Motion Perception.
J. Vacher, A.I. Meso, L.U. Perrinet, G. Peyr\'e -- Advances in Neural Information Processing Systems, 2015.}

% AM: I have also suggested some wording changes below... just minor ones, remove 'As such, ...' I also removed 'experimental' before neuroscience
We believe that this work presents a novel contribution of broad interest in neuroscience and psychophysics. We must stress that it combines the expertise in different fields for understanding neural computations: theoretical, computational and experimental. In addition, we demonstrate that a Bayesian model for vision is able to account for the effect of scene zooming, manipulated through spatial frequency on speed perception. We believe that these two contributions will be of significant interest to the readership of \Journal.


Following the change to the competing interests policy, we have revised the our ‘Conflict of interest’ statement to the following: 
‘
Competing interests: The authors declare no competing interests’
Thank you for your consideration, and we look forward to your response. 


Thank you for your consideration.%, and I look forward to hearing whether this article can be sent out for review.


Sincerely, on behalf of all co-authors, 

\end{newlfm}
\end{document}
